
\bfsection{2 Theory of heights}


\bfsubsection{Theorem 2.3}
\barquo{
We set $n = [K:\Q]$. Let $\{ \gro_1, \cdots , \gro_n \}$ be the integral basis of $O_K$. Then $\{ x\gro_1, \cdots , x\gro_n \}$ is a basis of V.
}
\begin{proof}
  There is a $c_{ij} \in \Z$ such that $x \gro_i = \sum_{j} c_{ij} \gro_i$. Set $C= (c_{ij}) \in M_n(\Z)$. Then $\det C = N_{K/\Q}(x) \neq 0 $, so we get $C \in GL_n(\Q)$. And we obtain the assertion.
\end{proof}

\bfsubsection{Proposition 2.5}
\barquo{
\[
h_K(x) \leq \sum_{\grs \in K(\C)} \log \left( \max_{1 \leq i \leq n} \{ \abs{x_i}_{\grs} \} \right).
\]
}
\begin{rem}
  \textblue{Misprint.} Add $1/[K:\Q]$ into the right.
\end{rem}

\bfsubsection{Proposition 2.6}
\barquo{
for any $x \in \ol{\Q}^n$.
}
\begin{rem}
  \textblue{Misprint.} Exclude the case $x = 0$.
\end{rem}




\bfsubsection{Proposition 2.8}
\barquo{
We consider two morphisms $\phi_1 \colon X \to \P^{m_1}$ and $\phi_2 \colon X \to \P^{m_2}$ over $\ol{\Q}$. If $\phi_1^* \calo_{\P^{m_1}} \cong \phi_2^* \calo_{\P^{m_2}}$, then there is a constant $C$ such that, for any $x \in X(\ol{\Q})$,
\[
\abs{ h_{\phi_1}(x) -  h_{\phi_2}(x) } \leq C.
\]
}
\begin{proof} ${}$
  Remark that $\calo_{\P^{m_1}}(1)$ is a Serre's twisted sheaf. See Bosch\cite{Bosch} 9.2/Definition 3. or Hartshorne\cite{ha} section 2.5 Adjacent to Proposition 5.12.
We set $L = \phi_1^* \calo_{\P^{m_1}}$ and set $k = \ol{\Q}$ and $\scrf = \calo_{\P^{m_1}}(1)$. Since $X$ is a projective variety over $k$ and $L$ is an invertible sheaf on $X$, so $H^0(X,L) = \Gamma(X,L)$ is a $k$-vector space of finite dimension. (See Hartshorne\cite{ha} section 2.5 Theorem 5.19. and Hartshorne\cite{ha} section 2.4 Prop 4.10)

Let $\{ t_0, \cdots , t_m \}$ be a basis of $H^0(X,L)$ and let $X_0, \cdots , X_{m_1}$ be the homogenous coordinates of $\P^{m_1}$. Note that each $X_i$ is a global section of $\scrf$. And we set $s_i = \phi_1^* X_i \in H^0(X,L)$, where $\phi_1^* X_i$ is the image of $X_i \in H^0(\P^m, \scrf)$ by the canonical map $\scrf \to {\phi_1}_* \phi_1^* \scrf$.
It follows from Hartshorne\cite{ha} section 2.7 Theorem 7.1 that $s_0, \cdots , s_{m_1}$ generate $L$. Because for any $x \in X$ the each germ $(s_i)_x$ is a linear combination of $(t_j)_x$, so $t_0, \cdots ,t_m$ generate $L$.

There is a morphism $\phi \colon X \to \P^m $ such that $L \cong \phi^* \calo_{\P^m}(1)$ and $s_i = \phi^* X_i$ under this isomorphism. See Hartshorne\cite{ha} section 2.7 Theorem 7.1(b). There is another explanation on what $\phi$ is. For any $x \in X$, we can consider the germ $(t_i)_x \in L_x$. Denote $(t_i)_x$ by $t_i(x)$. Since $L$ is a line bundle, $L_x \cong k$. (\textblue{It remains to be learned.})
Then we define the map $\phi \colon X \to \P^m$ by $\phi(x) = (t_0(x), \cdots , t_m(x))$. Note that there is a scalar ambiguity in choice of isomorphism $L_x \to k$. If $\forall i \; t_i(x) = 0$, then $(t_i)_x$ cannot generate $L_x$, which is a contradiction. Thus for any $x \in X$,
there is an index $i$ such that $t_i(x) \neq 0$.

Then, the rest of the proof is almost trivial.
\end{proof}
