
\bfsection{2 Theory of heights}

\begin{comment}%ただの疑問だから、Stackに投稿すべきかな~
\bfsubsection{Adjacent to 2.4}
\barquo{
In other words, the absolute (logarithmic) Weil height is not invariant under linear coordinate changes. This is why it is sometimes called the naive height.
}
\begin{que}
  Why they want the height to be invariant? Is there a non-constant invariant height?
\end{que}
\end{comment}

\bfsubsection{Proposition 2.8}
\barquo{
If $\phi_1^*(O_{\P^{m_1} }(1)  ) \cong \phi_2^*(O_{\P^{m_2} }(1)  ) $,
}
\begin{rem}
What is a $O_{\P^{m_1} }(1)$? I think it is a Serre's twisted sheaf. See Bosch\cite{Bosch} 9.2/Definition 3. \textblue{It remains to be learned.}
\end{rem}
