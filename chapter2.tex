
\bfsection{2 Theory of heights}


\bfsubsection{Theorem 2.3}
\barquo{
We set $n = [K:\Q]$. Let $\{ \gro_1, \cdots , \gro_n \}$ be the integral basis of $O_K$. Then $\{ x\gro_1, \cdots , x\gro_n \}$ is a basis of V.
}
\begin{proof}
  There is a $c_{ij} \in \Z$ such that $x \gro_i = \sum_{j} c_{ij} \gro_i$. Set $C= (c_{ij}) \in M_n(\Z)$. Then $\det C = N_{K/\Q}(x) \neq 0 $, so we get $C \in GL_n(\Q)$. And we obtain the assertion.
\end{proof}

\bfsubsection{Proposition 2.5}
\barquo{
\[
h_K(x) \leq \sum_{\grs \in K(\C)} \log \left( \max_{1 \leq i \leq n} \{ \abs{x_i}_{\grs} \} \right).
\]
}
\begin{rem}
  \textblue{Misprint.} Add $1/[K:\Q]$ into the right.
\end{rem}

\bfsubsection{Proposition 2.6}
\barquo{
for any $x \in \ol{\Q}^n$.
}
\begin{rem}
  \textblue{Misprint.} Exclude the case $x = 0$.
\end{rem}




\bfsubsection{Proposition 2.8}
\barquo{
We consider two morphisms $\phi_1 \colon X \to \P^{m_1}$ and $\phi_2 \colon X \to \P^{m_2}$ over $\ol{\Q}$. If $\phi_1^* \calo_{\P^{m_1}} \cong \phi_2^* \calo_{\P^{m_2}}$, then there is a constant $C$ such that, for any $x \in X(\ol{\Q})$,
\[
\abs{ h_{\phi_1}(x) -  h_{\phi_2}(x) } \leq C.
\]
}
\begin{proof} ${}$
  Remark that $\calo_{\P^{m_1}}(1)$ is a Serre's twisted sheaf. See Bosch\cite{Bosch} 9.2/Definition 3. or Hartshorne\cite{ha} section 2.5 Adjacent to Proposition 5.12.
We set $L = \phi_1^* \calo_{\P^{m_1}}$ and set $k = \ol{\Q}$ and $\scrf = \calo_{\P^{m_1}}(1)$. Since $X$ is a projective variety over $k$ and $L$ is an invertible sheaf on $X$, so $H^0(X,L) = \Gamma(X,L)$ is a $k$-vector space of finite dimension. (See Hartshorne\cite{ha} section 2.5 Theorem 5.19. and Hartshorne\cite{ha} section 2.4 Prop 4.10)

Let $\{ t_0, \cdots , t_m \}$ be a basis of $H^0(X,L)$ and let $X_0, \cdots , X_{m_1}$ be the homogenous coordinates of $\P^{m_1}$. Note that each $X_i$ is a global section of $\scrf$. And we set $s_i = \phi_1^* X_i \in H^0(X,L)$, where $\phi_1^* X_i$ is the image of $X_i \in H^0(\P^m, \scrf)$ by the canonical map $\scrf \to {\phi_1}_* \phi_1^* \scrf$.
It follows from Hartshorne\cite{ha} section 2.7 Theorem 7.1 that $s_0, \cdots , s_{m_1}$ generate $L$. Because for any $x \in X$ the each germ $(s_i)_x$ is a linear combination of $(t_j)_x$, so $t_0, \cdots ,t_m$ generate $L$.

There is a morphism $\phi \colon X \to \P^m $ such that $L \cong \phi^* \calo_{\P^m}(1)$ and $s_i = \phi^* X_i$ under this isomorphism. See Hartshorne\cite{ha} section 2.7 Theorem 7.1(b). There is another explanation on what $\phi$ is. For any $x \in X$, we can consider the germ $(t_i)_x \in L_x$. Denote $(t_i)_x$ by $t_i(x)$. Since $L$ is a line bundle, $L_x \cong \calo_{X,x}$.
Then we define the map $\phi \colon X \to \P^m$ by $\phi(x) = (t_0(x), \cdots , t_m(x))$. Note that there is a scalar ambiguity in choice of morphism $L_x \to k$. If $\forall i \; t_i(x) = 0$, then $(t_i)_x$ cannot generate $L_x$, which is a contradiction. Thus for any $x \in X$,
there is an index $i$ such that $t_i(x) \neq 0$.

Then, the rest of the proof is almost trivial.
\end{proof}


\bfsubsection{Theorem 2.9}
\barquo{
First, suppose that $L$ is globally generated.
}
\begin{rem}
  What "globally generated" means? We say $L$ is globally generated iff there is an exact sequence $\bigoplus_I O_X \to L \to 0$. Even if $L$ is an invertible sheaf, $L$ is not necessarily globally generated. For example, set $X = \P^m$, $L = \calo_{\P^m}(-1)$. Since $\Gamma(X,L) = 0$, $L$ is not globally generated.
\end{rem}


\bfsubsection{Theorem 2.9}
\barquo{
Let
\[
\phi_{\abs{L}} \colon X \to \P(H^0(X,L))
\]
be a morphism associated to the complete linear system $\abs{L}$. We set $h_L = h_{\phi_{\abs{L}}}$.
}
\begin{proof}
Note that we want to get $h_L \in \Func(X)/B(X)$, which is not contained in  $\Func(X)$.

What is a $\P(H^0(X,L))$? I think it is isomorphic to $\P^m$ by taking a basis of $H^0(X,L)$.

  Set $k = \ol{\Q}$. Since $L$ be a globally generated line bundle on $X$, there is a basis $s_0, \cdots , s_m$ of $H^0(X,L)$ which generate $L$. Then we get a map $\phi_L \colon X \to \P^m$ such that $\phi_L^* \calo_{\P^m}(1) \cong L$. We define $h_L$ by $h_L = h_{\phi_L}$.
\end{proof}


\bfsubsection{Theorem 2.9}
\barquo{
Then ${s_i \ts t_j} $ induces a morphism $\phi \colon X \to \P^N$ such that $\phi^*(O_{\P^N(1)}) \cong L_1 \ts L_2$.
}
\begin{rem}
  How $s_i \ts t_j \in H^0(X,L_1) \ts_k H^0(X,L_2)$ define an element of $H^0(X,L_1 \ts L_2)$? Let $\scrf$ be a presheaf defined by $\scrf(U) = \grG(U,L_1) \ts_{\calo_{X}(U)} \grG(U,L_2)$. Then there is a canonical morphism $\scrf \to L_1 \ts L_2$ since $L_1 \ts L_2$ is the sheafification of $\scrf$.
  So we can see $s_i \ts t_j \in H^0(X,L_1 \ts L_2)$.

  We denote the image of $s_i \ts t_l$ by $s_it_j \in H^0(X,L_1 \ts L_2)$. Why $\{ s_it_j \}$ generate $L_1 \ts L_2$? Take a stalk.
\end{rem}


\bfsubsection{Theorem 2.9}
\barquo{
tell us that $L \ts A^n$ is globally generated for any sufficiently large $n$.
}
\begin{rem}
  The ampleness of $A$ implies that
  \begin{align*}
    \exists n_1 \st n \geq n_1 &\To L \ts A^n \; \text{is globally generated} \\
      \exists n_2 \st n \geq n_2 &\To  A^n \; \text{is globally generated}
  \end{align*}
  Then we set $n = \max_i \{ n_i\}$.
\end{rem}


\bfsubsection{Theorem 2.9}
\barquo{
Then, modulo $B(X)$, we have
\[
h_{f^*(L)} = h_{f^*(C) \ts f^*(C)^{-1}}
\]
}
\begin{rem}
  See G\"ortz Wedhorn\cite{GW} Remark 7.10.
\end{rem}


\bfsubsection{Theorem 2.9}
\barquo{
Then by (1), $h_L$ must be equal to $h_{L_1} - h_{L_2}$ modulo $B(X)$.
}
\begin{rem}
  Let $\grs \colon \{ \text{line bundles }\} \to \Func(X)/B(X)$ be a map which satisfies the properties (1), (2), (3). By (3), for globally generated line bundle $L$, we get $\grs_L = h_L$. Because $\grG(X,O_X) = k$, we obtain $\grs_{O_X} = 0$. Thus (1) implies that $\grs_L = h_L$ for general line bundle $L$.
\end{rem}


\bfsubsection{Proposition 2.10}
\barquo{
Let $B$ be the Zariski closed subset of $X$ defined by the ideal sheaf
\[
\Im (H^0(X,L) \ts L^{-1} \to O_X).
\]
}
\begin{rem}
  What is the morphism $H^0(X,L) \ts L^{-1} \to O_X$? Note that there is a canonical morphism $f^* f_* L \to L$ where $f \colon X \to \Spec k$ is a $k$-scheme structure. Note that $f_* L$ is isomorphic to $\wt{H^0(X,L)}$. We denote this canonical morphism $f^* f_* L \to L$ by
  \[
  H^0(X,L) \ts O_X \to L.
  \]
  This is surjective if $L$ is globally generated.

  In general, we define $V \ts_k O_X$ for $k$-module $V$, by setting
  \[
  V \ts_k O_k = f^{-1} \wt{V} \ts_{f^{-1}O_{\Spec k}} O_X = f^* \wt{V}.
  \]
Then we get $H^0(X,L) \ts L^{-1} \to O_X$ by tensoring $L^{-1}$.
\end{rem}


\bfsubsection{Proposition 2.10}
\barquo{
Then $\{  s s_i \}$ are linearly independent elements of $H^0(X,L)$.
}
\begin{rem}
  What are $s s_i$ ? Note that there is a canonical morphism
  \[
  H^0(X,L) \ts H^0(X,L_2) \to H^0(X,L \ts L_2) \cong H^0(X,L_1)
  \]
  Thus I guess $s s_i$ is the image of $s \ts s_i$.

  Moreover, why $s s_i$ are linearly independent? It suffices to show that the morphism of $k$-module
  \[
  s \colon H^0(X,L_2) \to H^0(X,L_1)
  \]
  is injective.

  We prepare the following lemma.
  \lem{
  Let $X$ be an integral scheme and let $L$ be a line bundle on $X$. Assume that $s \in H^0(X,L)$ is not zero. Then for any $x \in X$, $s_x \neq 0$ in $L_x$.
  }
  \begin{proof}
    Assume that there is a $z \in X$ such that $s_z = 0$. We want to show $s = 0 \in H^0(X,L)$. Since  $L$ is invertible, there is an open affine covering $X = \bigcup_{i \in I} U_i$ such that
    \[
    U_i = \Spec A_i , \; L|_{U_i} \cong \wt{A_i}
    \]
    On the other hand, $s_z = 0$ implies that there is an open subset $U \subset X$ such that $s|_U = 0$ and $z \in U$. Since $X$ is integral, $U \cap \Spec A_i \neq \emptyset$. Thus there is a $g_i \in A_i \sm \{ 0 \}$ such that $\emptyset \neq D(g_i) \subset U \cap \Spec A_i$. Then $s|_{D(g_i)} = 0$ in $\grG(D(g_i), L) \cong {A_i}_{g_i}$.
    Note that each $A_i$ is an integral domain because $X$ is integral. Thus we get $\forall i \; s|_{U_i}=0$ because $A_i \to {A_i}_{g_i}$ is injective. It follows from the sheaf axiom that $s=0 \in \grG(X,L)$.
  \end{proof}

  Then, we can prove the injectivity of $s \colon O_X \to L$. First, by the lemma, $0 \to O_X \xrightarrow{s} L$ is exact. Since $L_2$ is flat, $0 \to L_2 \xrightarrow{s} L_1$ is exact. Since global section is left exact, we get $0 \to H^0(X,L_2) \xrightarrow{s} H^0(X,L_1)$ is exact.
\end{rem}



\bfsubsection{Proposition 2.10}
\barquo{
Let $s_1, \cdots , s_n$ be a basis of $H^0(X,L)$. $\cdots$

 Because $B = \setmid{x \in X}{s_1(x) = \cdots = s_n(x) = 0}$.
}
\begin{proof}
  Why $B = \setmid{x \in X}{s_1(x) = \cdots = s_n(x) = 0}$? I guess
  \begin{align*}
    B &= \Supp \Coker(H^0(X,L) \ts L^{-1} \to O_X) \\
    &= \Supp \Coker(H^0(X,L) \ts O_X \to L) \ts L^{-1}  &(\text{right exactness of tensor}) \\
    &= \Supp \Coker(H^0(X,L) \ts O_X \to L) \cap \Supp L^{-1} \\
    &= \Supp \Coker(H^0(X,L) \ts O_X \to L)
  \end{align*}
  Thus we get
  \begin{align*}
    x \in B &\iff \Coker(H^0(X,L) \ts O_X \to L)_x \neq 0 \\
    &\iff \forall s \in H^0(X,L) \; x_x \in \frakm_x L_x \\
    &\iff s(x) = 0 \\
    &\iff s_1(x) = \cdots = s_n(x) = 0
   \end{align*}
\end{proof}

\begin{comment}
\bfsubsection{Lemma 2.21}
\barquo{



LEMMA2.21. $\;$ Let $X$ be a projective variety, and let $x_0$ and $x_1$ be closed points of $X$. Then there is a projective curve $C$ on $X$ passing through $x_0$ and $x_1$.

PROOF. If $x_0 = x_1$, then the result is obvious, so we assume that $x_0 \neq x_1$. We show the lemma by induction on dimension of $X$. The dimension one case is obvious. Suppose $\dim X \geq 2$ and let $\pi \colon \tilde{X} \to X$ be the blowup of $X$ along the two points $x_0,x_1$. We embed $\tilde{X}$ into projective space $\P^N$. We take $N$ to be the minimum among such embeddings. For a general hyperplane $H \subset \P^N$,
$\tilde{Y} = \tilde{X} \cap H$ is irreducible and is not contatined in $\pi^{-1}(x_i)$.(See [Hartshorne section.2 theorem8.18])
Further, because $\dim H + \dim \pi^{-1}(x_i) \geq N$, $H \cap \pi^{-1}(x_i) \neq \emptyset$. (See [Hartshorne section.1 theorem 7.2]) If we set $Y = \pi(\tilde{Y}) \subset X$, then $Y$ is a $(\dim X - 1)$-dimensional subvariety of $X$ that contains $x_0$ and $x_1$. By the induction hypothesis, there is a projective curve $C$ on $Y$ that passes through $x_0$ and $x_1$.
This $C$ is a desired curve.
}
\begin{que} ${}$
  \begin{description}
    \item[(1)] 「If $x_0 = x_1$, then the result is obvious」なのはどうしてでしょうか。その点を通る多項式$f$をとって$X \cap V(f)$を考えるとふつうは一つ次元が下がりますが、これを繰り返せばよいということでしょうか。
    \item[(2)] 「We embed $\tilde{X}$ into projective space $\P^N$.」とありますが、$X$が射影的な多様体なら$\tilde{X}$も射影的というHartshorne section 2 命題7.16から従っているのでしょうか。
    \item[(3)] 「$\tilde{Y} = \tilde{X} \cap H$ is irreducible and is not contatined in $\pi^{-1}(x_i)$.」とあります。引用されているBertiniの定理(とHartshorne section 3 注意7.9.1)から既約性がわかるのはいいとして、「and is not contatined in $\pi^{-1}(x_i)$.」はなぜわかるのでしょうか。
    \item[(4)] 「Further, because $\dim H + \dim \pi^{-1}(x_i) \geq N$, $H \cap \pi^{-1}(x_i) \neq \emptyset$」とあります。前半から後半が射影次元定理で従うのはよいです。前半はなぜ判るのでしょうか。
    \item[(5)] 「If we set $Y = \pi(\tilde{Y}) \subset X$, then $Y$ is a $(\dim X - 1)$-dimensional subvariety of $X$ that contains $x_0$ and $x_1$.」とあります。なぜ次元がこうなるのでしょうか。
  \end{description}
\end{que}



\newpage


\bfsubsection{Theorem 2.23}
\barquo{
Theorem 2.23 (Theorem of the cube II)

Let $X,Y$, and $Z$ be geometrically irreducible algebraic varieties over $F$, let $x_0 \in X(F)$, $y_0 \in Y(F)$, and $z_0 \in Z(F)$ be closed points, and let $L$ be a line bundle on $X \times Y \times Z$. Assume that $X$ and $Y$ are projective over $F$. Then $L$ is trivial if and only if
\[
 L|_{\{x_0\} \times Y \times Z}, \; L|_{X \times \{y_0\} \times Z}, \; L|_{X \times Y \times \{z_0\}}
\]
are all trivial.


\textbf{proof of  theorem 2.23}
\begin{description}
  \item[Step 1] Let $p_{13}$ denote the projection $X \times Y \times Z \to X \times Z$ to the first and the last factor. If we show that $L|_{ \{x\} \times Y \times \{z\} }$ is trivial for every closed points $x \in X$ and $z \in Z$, then, by the seesaw theorem, we can find a line bundle $M$ on $X \times Z$ such that $p_{13}^* M \cong L$. Further,
  $L|_{X \times \{y_0\} \times Z}$ is trivial
  by the hypothesis, so $M$ is trivial. Thus $L$ is trivial and we will obtain the theorem.
  \item[Step 2] $\;$ For any closed point $x \in X$, lemma 2.21 tells us that there is a projective cureve $C$ on $X$ passing $x_0$ and $x_1$. Let $\tilde{C} \to C$ be the normalization of $C$. Suppose that we show Theorem 2.23 for $X= \tilde{C}$. Then $L|_{\tilde{C} \times Y \times Z}$ will be trivial,
  so $L|_{ \{x\} \times Y \times \{z\} }$ will be trivial for any closed point $z \in Z$.
  As $x$ is an arbitrary closed point of $X$, by Step 1, we will obtain Theorem 2.23 for general $X$.
  \item[Step 3] By Step 2, we may assume that $X$ is a smooth projective curve. To conclude the proof, we here use some properties of the Jacobian variety $J$ of $X$, which we explain in Section 2.8 below. By [Hartshorne"Algebraic Geometry" section3 -9.9], $L$ has degree zero on each fiber of $X \times (Y \times Z) \to Y \times Z$. Thus $L$ defines a morphism
  \[
  f \colon Y \times Z \to J.
  \]
  Because both $\{y_0\} \times Z$ and $Y \times \{z_0\}$ are mapped to $0$ by $f$, the rigidity lemma tells us that the whole $Y \times Z$ is also mapped to $0$ by $f$. Thus $L$ is trivial on $X \times Y \times Z$.
\end{description}
}

\begin{que} ${}$
  \begin{description}
    \item[(1)]「To conclude the proof, we here use some properties of the Jacobian variety $J$ of $X$,」とありますが、$X$のヤコビ多様体とはなんでしょうか。(定義がわかりません)そしてここで使う性質とは何でしょうか。
    \item[(2)] 「$L$ has degree zero on each fiber of $X \times (Y \times Z) \to Y \times Z$. 」とあります。これはどうしてでしょうか。
    Hartshorneの引用された定理を見ると、Hilbert多項式がからんでいるようなのですが、そこから先に進めません。
    \item[(3)] 「Thus $L$ defines a morphism $f \colon Y \times Z \to J.$」とありますが、これはなぜでしょうか。どういう射でしょうか。
  \end{description}
  以上です。
\end{que}
\end{comment}

\bfsubsection{Cor 2.25}
\barquo{
Let $A$ be an abelian variety of dimension $g$ over $F$.
\begin{description}
  \item[(1)] For any integer $n$, $[n] \colon A \to A$ is a finite and flat morphism of degree $n^{2g}$.
  \item[(2)] The abelian group $A(\ol{F})$ is divisible, i.e., for any $x \in A(\ol{F})$ and for any positive integer $n$, there is a $y \in A(\ol{F})$ with $[n](y) = x$.
\end{description}
}
\begin{proof} ${}$
  \begin{description}
    \item[(1)] $L$をevenかつampleなline bundleとする。$[n]^* L = L^{n^2}$より$[n]^* L$もampleである。$[n]^*L$がampleということは、(閉とは限らない)埋め込み
    \[
    \psi_{\abs{\wt{L}}} \colon A \to \P(\grG (A, \wt{L})^{\vee}  )
    \]
    がある。ただし$\wt{L} = [n]^* L$であり、${}^{\vee}$は双対空間を表す。このとき次の図式は可換。
    \[
    \xymatrix{
    A  \ar[r]^-{\psi_{\abs{\wt{L}}}} & \P(\grG (A, \wt{L})^{\vee}  ) \\
    \Ker [n] \ar[u] \ar[r]^-{\psi_{\abs{ \Ker [n] }  }} & \P(\grG (\Ker [n], \wt{L}|_{\Ker [n] } )^{\vee}  ) \ar[u]
    }
    \]
    ここで$\wt{L}|_{\Ker [n] }$は自明なので$\Ker [n]$の既約性分への分解を$\Ker [n] = \coprod_{i \in I} P_i$とすると、各成分$P_i$の$\psi_{\abs{ \Ker [n] }} $による像は$\P^{0}(F)$に含まれる。つまり一点である。
    したがって、$\psi_{\abs{\wt{L}}}$は埋め込みなので各$P_i$は一点である。
    よって$\dim \Ker [n] =  0$である。

    また$[n]$はprojective varietyの間の射なのでprojectiveであり、とくに固有である。$\dim \Ker [n] =  0$であることをいま示したが、$[n]$は準同型なのですべての点のfiberの次元が等しい。ゆえに固有かつすべての点でのfiberの次元がゼロなので$[n]$はfiniteである。とくにfiberの次元が任意の点で等しいので$[n]$はflatである。

    次にdegreeについて考える。ここでのdegreeは交点数を用いて定義される。ample line bundle $L$について
    \[
    L^{\cdot g} = (L. \cdots . L)
    \]
    と定義する。右辺の$L$は$g$個ある。(Reference: 石井志保子「特異点入門」)交点数は多重線形なので
    \[
    ( [n]^* L)^{\cdot g} = (L^{n^2}. \cdots . L^{n^2}) = n^{2g} L^{\cdot g}
    \]
    が従う。よってdegreeの定義から$\deg [n] = n^{2g}$である。
    \item[(2)] $[n] \colon A \to A$は固有なのでその像$[n](A)$は閉部分多様体である。また$[n]$はflatなので
    \[
    \dim_0 A + \dim_0 \Ker [n] = \dim_0 [n](A)
    \]
    が成り立つ。よって$\dim \Ker [n] =  0$より$\dim A = \dim_0 A = \dim_0 [n](A) = \dim [n](A)$である。真閉部分多様体は次元が落ちるはずなので$A = [n](A)$がわかる。
  \end{description}
\end{proof}


\bfsubsection{Remark 2.27}
\barquo{
If $D$ is an effective Cartier divisor on an abelian variety $A$, then $\abs{2D}$ is base point free. In particular, $D$ is nef.
}
\begin{rem}
  まず用語について解説する。$D$がbase point freeとは、rational map
  \[
  \psi_{\abs{D}} \colon A \dashrightarrow \P ( \grG (A, D)^{\vee} )
  \]
  が$A$全体で定義されることである。言い換えれば、
  \[
  \setmid{x \in A}{ \forall  0 \neq s \in \grG(A,D) \; s(x)=0 } = \emptyset
  \]
  ということである。また$D$がnef(数値的正、ネフ)とは
\[
\forall C \text{ irreducible curve} \; (D.C) \geq 0
\]
(交点数がゼロ以上)として定義される。

さてeffectiveなCartier divisor $D$について、$2D$がbase point freeならば$D$がnefであることを確かめよう。$2D$がnefならあきらかに$D$もnefなので、はじめから$D$がbase point freeだと仮定して$D$がnefだといえばよい。

いま$D$はeffectiveなので既約曲線$C$が$C \not\subset D$である限り、$(D.C) \geq 0$となる。交点数は線形同値なもの同士を入れ替えても不変なので、どんな$C$についてもある$D$と線形同値な$D'$があって$C \not\subset D'$となることをいえばよい。

ハイリホーで示す。ある既約曲線$C$が存在して、すべての$D' \sim D$なる$D'$について$C \subset D'$であったとする。いま$\psi_{\abs{D}}$の定義から、任意の超曲面$H \subset \P ( \grG (A, D)^{\vee} )$に対して$D \sim \psi_{\abs{D}}^* H$である。$H$はある大域切断$s \in \grG(A,D)$により$H = \setmid{l}{l(s) = 0}$と表せる。
そこでこれを$H_s$とおく。このとき仮定から
\[
\forall s \; C \subset \psi_{\abs{D}}^* H_s
\]
であるが、$\psi_{\abs{D}}^* H_s = \setmid{x \in A}{s(x)=0}$であったため、$D$がbase point freeであったことより$C = \emptyset$となるしかない。これは矛盾である。
\end{rem}


\newpage

\bfsubsection{Remark 2.27}

\barquo{
For an $a \in A(\ol{F})$, we define the morphism $T_a \colon A_{\ol{F}} \to \Pic(A_{\ol{F}} )$ by
\[
T_a \colon x \mapsto x + a.
\]
For any given line bundle $L$ on $A_{\ol{F}}$, we define the map $\grl_L \colon A(\ol{F}) \to \Pic( A_{ \ol{F} } )$ by
\[
\grl \colon x \mapsto T_x^*L \ts L^{-1}.
\]

Cor (Theorem of Square)

As groups, the map $\grl_L$ is a homomorphism from $A(\ol{F})$ to $\Pic( A_{ \ol{F} } )$.
}

\barquo{
If $D$ is an effective Cartier divisor on an abelian variety $A$, then $\abs{2D}$ is base point free. In particular, $D$ is nef. Indeed, identifying $D$ with the corresponding Weil divisor, we write $D + a$ and $D-a$ for $T_a(D)$ and $T_{-a}(D)$ respectively. For any $x \in A(\ol{F})$, we choose a point $a \not\in \Supp (D-x) \cup \Supp(D +x)$. Then $x \not\in \Supp (D-a) \cup \Supp(D +a)$. Further, the theorem of the square implies that $(D-a) + (D+a) \sim 2D$.
Thus $\abs{2D}$ is base point free.
}




\begin{rem}
Hartshorne\cite{ha} section 2.6 命題6.11の、$X$が整分離的Noetherスキームで局所分解的なものとすると、Cartier divisorとWeil因子が同型になるということを踏まえて同一視をする。ここでAbelian多様体$A$はsmoothで、したがって局所環がregularであり、正則局所環はUFDであることから$A$は局所分解的であることに気を付ける。

閉点$x \in A(\ol{F})$が任意に与えられたとする。$x$を台に含まないような、$2D$と線形同値なeffective因子の存在をいえば、$\abs{2D}$がbase point freeであることが従う。そういう因子を構成しよう。

点$a \not\in \Supp(D-x) \cup \Supp(-D + x)$をとる。ここで因子$D-x$は像$T_{-x}(D)$を意味し、$-D + x$は逆元をとる写像を$i$として$T_x(i (D))$を意味する。このとき$a \not\in \Supp(D-x)$より$x \not\in \Supp(D-a)$であり、$a \not\in \Supp(-D+x)$より$x \not\in \Supp(D + a)$である。(本文には\textblue{誤植がある})このときTheorem of squareにより
\begin{align*}
  (D-a) + (D+a) &= T_{-a}(D) + T_{a}(D) \\
  &= \grl_{D}(-a) + \grl_D(a) + 2D \\
  &= 2D
\end{align*}
だから$2D$と$(D+a) + (D-a)$は線形同値。$(D+a) + (D-a)$はあきらかにeffectiveなので$\abs{2D}$がbase point freeであることがいえた。



\end{rem}


\newpage

\bfsubsection{Cor 2.28}
\barquo{
Let $A$ be an abelian variety, and let $D$ be an effective Cartier divisor on $A$. We set $L = O_X(D)$. (In particular, $\abs{2D}$ is base point free) Then the following are equivalent.
\begin{description}
  \item[(1)] $\Ker (\grl_L)$ is a finite subgroup of $A(\ol{F})$.
  \item[(2)] A morphism $\Phi \colon A \to \P(H^0(A,L^2))$ associated to the complete linear system $\abs{2D}$ is a finite morphism onto its image.
  \item[(3)] $L$ is ample.
\end{description}

PROOF.
Properties (2) and (3) hold over $F$ if and only if those hold over $\ol{F}$, so we may assume that $F = \ol{F}$.

\begin{description}
  \item[(1) $\To$ (2)] Suppose that $\Phi$ maps a projective curve $C$ on $A$ to a single point. We are going to deduce a contradiction by showing that $\Ker (\grl_L)$ contains $C -C= \setmid{x_2 - x_1}{x_1, x_2 \in C(F)}$. We write $D = \sum_{i=1}^r a_i D_i$ with $a_i > 0$ and prime divisors $D_i$ on $A$.

  We claim that either $(C + x) \cap D_i = \emptyset $ or $C + x \subset D_i$ for any $i$ and for any $x \in A(F)$. Because $D_i$ is an effective Cartier divisor on an abelian variety, it is nef by Remark 2.27. In particular, $(D_i \cdot C) \geq 0$. It follows from
  \[
  0 = (D \cdot C) = a_1 (D_1 \cdot C) + \cdots + a_r (D_r \cdot C),
  \]
  that $(D_i \cdot C) = 0$. Further, because $C + x$ is algebraically equivalent to $C$, we have $(D_i \cdot C + x) =(D_i \cdot C)=0$. Thus we obtain the claim.

  Next, we claim that $D_i = D_i + x_1 - x_2$ for any $x_1, x_2 \in C(F)$. Indeed, if $y \in D_i$ is a closed point, then both $C - x_1 + y$ and $D_i$ contain $y$. Thus $C - x_1 + y \subset D_i$ by the above argument. It follows that $x_2 - x_1 + y \in D_i$, so $y \in D_i + x_1 - x_2$. Thus $D_i \subset D_i + x_1 - x_2$. By switching $x_1$ and $x_2$ in the above argument, we have $D_i \supset D_i + x_1 - x_2$. Hence we obtain the claim.

  We set $L_i = O_A(D_i)$. Then $L = L_1^{\ts a_1} \ts \cdots \ts  L_r^{\ts a_r}$. By the above claim, we have $T_{x_2 - x_1}^*(L_i) = L_i$ for each $i$. It follows that
  \begin{align*}
    T_{x_2 - x_1}^*(L) &= T_{x_2 - x_1}^*(L_1)^{\ts a_1} \ts \cdots \ts T_{x_2 - x_1}^*(L_r)^{\ts a_r} \\
    &= L_1^{\ts a_1} \ts \cdots \ts  L_r^{\ts a_r} \\
  &= L.
  \end{align*}
  Thus $x_2 - x_1 \in \Ker (\grl_L)$, and so $C -C \subset \Ker (\grl_L)$. This is a contradiction.

  \item[(2)$\To$(3)] Let $E$ be any coherent $O_A$-module on $A$. We are going to show that $E \ts L^{2n}$ is globally generated for any sufficiently large $2n$. We write $X$ for the image $\Phi(A)$ of $\Phi$. Because $\Phi_*(E)$ is coherent on $X$ and $O_X(1)$ is ample, there is an $n_0 > 0$ such that
  \[
  H^0(X, \Phi_*(E) \ts O_X(n) ) \ts O_X \to \Phi_*(E) \ts O_X(n)
  \]
  is surjective for any $n \geq n_0$ (See Hartshorne p.153). Pulling back by $\Phi$, we find that
  \[
  (*)  \quad  \quad    H^0(X, \Phi_*(E) \ts O_X(n) ) \ts O_A \to \Phi^* \Phi_*(E) \ts \Phi^* O_X(n)
  \]
  is also surjective.

  On the other hand, it follows from $\Phi^* O_X(n) \cong L^{2n}$ and the projection formula that $\Phi_*(E \ts L^{2n}) \cong \Phi_*(E) \ts O_X(n)$. Further, as $\Phi$ is a finite morphism, the canonical morphism $\Phi^* \Phi_*(E) \to E$ is surjective. Thus the surjectivity of $(*)$ implies the surjectivity of
  \[
  H^0(A, E \ts L^{2n} ) \ts O_A \to E \ts L^{2n}
  \]
  for every $n \geq n_0$. Thus $L^2$ is ample, so $L$ is ample.

  \item[(3)$\To$(1)] Let $p_i \colon A \tm A \to A$ denote the i-th projection. First, we note the equality
  \[
  \Ker \grl_L = \setmid{ x \in A(F)  }{ (m_A^* L^{-1} \ts p_1^* L)|_{A \tm \{ x \} } \text{ is trivial}  }.
  \]
  In particular, the seesaw theorem tells us that $\Ker \grl_L$ endowed with the reduced induced scheme structure is regarded as a closed subgroup scheme of $A$. We denote by $B$ the connected component of $\Ker(\grl_L)$ containing the identity. Then $B$ is an abelian subvariety of $A$. We are going to show that $\dim B = 0$.

  We set $L' := (m_A^* L^{-1} \ts p_1^* L \ts p_2^* L)|_{B \tm B}$ on $B \tm B$. Because $L'|_{B \tm \{x\}} = (m_A^* L^{-1} \ts p_1^* L)|_{B \tm \{x\}} \ts p_2^* L|_{B \tm \{x\}}$ is trivial for any $x \in B$
  and $L'|_{ \{0\} \tm B}$ is trivial, the seesaw theorem implies that $L'$ is trivial. Pulling back $L'$ by $([1]_B, [-1]_B) \colon B \to B \tm B$, we obtain that $L \ts [-1]_A^* L $ is trivial on $B$. On the other hand, because $L$ is ample on $A$, $L \ts [-1]_A^* L$ is ample on $A$. Thus $\dim B = 0$,
   and we conclude that $\Ker(\grl_L)$ is a finite set.
  \end{description}

}
\begin{que} ${}$
\begin{description}
  \item[(1)] 「Properties (2) and (3) hold over $F$ if and only if those hold over $\ol{F}$」とあります。これは何故なのでしょうか。
  \item[(2)] (1)$\To$(2)の証明で、背理法(対偶)が使われているようです。(2)が成り立たず、$\Phi$がfiniteでなかったとして、$\Ker \grl_L$が無限集合であることを示す証明になっているように見えます。そこで$\Phi$がfiniteでなければある$x$があって、$\Phi^{-1}(x)$に含まれるprojectiveな曲線$C$があるということを使っているようですが、どうしてそういう$C$が存在するのでしょうか。

  $\Phi$のすべてのファイバーの次元が$0$だとすると、$\Phi$はproperなのでfiniteになります。よって$\Phi$がfiniteでないという仮定から、$\Phi$のある点でのファイバーの次元は$1$以上であり、したがってあるファイバーに含まれるような射影曲線$C$がとれる、という証明で問題ないですか?
  \item[(3)] 「Further, because $C + x$ is algebraically equivalent to $C$」とあります。algebraically equivalentとはどういう意味(定義)で、そしてどうしてそれが分かるのでしょうか。
  \item[(3)]
\end{description}

\end{que}
