
\bfsection{2 Theory of heights}


\bfsubsection{Theorem 2.3}
\barquo{
We set $n = [K:\Q]$. Let $\{ \gro_1, \cdots , \gro_n \}$ be the integral basis of $O_K$. Then $\{ x\gro_1, \cdots , x\gro_n \}$ is a basis of V.
}
\begin{proof}
  There is a $c_{ij} \in \Z$ such that $x \gro_i = \sum_{j} c_{ij} \gro_i$. Set $C= (c_{ij}) \in M_n(\Z)$. Then $\det C = N_{K/\Q}(x) \neq 0 $, so we get $C \in GL_n(\Q)$. And we obtain the assertion.
\end{proof}

\bfsubsection{Proposition 2.5}
\barquo{
\[
h_K(x) \leq \sum_{\grs \in K(\C)} \log \left( \max_{1 \leq i \leq n} \{ \abs{x_i}_{\grs} \} \right).
\]
}
\begin{rem}
  \textblue{Misprint.} Add $1/[K:\Q]$ into the right.
\end{rem}

\bfsubsection{Proposition 2.6}
\barquo{
for any $x \in \ol{\Q}^n$.
}
\begin{rem}
  \textblue{Misprint.} Exclude the case $x = 0$.
\end{rem}

\bfsubsection{Proposition 2.8}
\barquo{
If $\phi_1^*(O_{\P^{m_1} }(1)  ) \cong \phi_2^*(O_{\P^{m_2} }(1)  ) $,
}
\begin{rem}
What is a $O_{\P^{m_1} }(1)$? I think it is a Serre's twisted sheaf. See Bosch\cite{Bosch} 9.2/Definition 3.
\end{rem}


\bfsubsection{Proposition 2.8}
\barquo{
Let $\{ t_0, \cdots , t_m \}$ be a basis of $H^0(X,L)$, and we set $\phi = (t_0 : \cdots : t_m)$.
}
\begin{rem} ${}$
  \begin{description}
    \item[Why $H^0(X,L)$ is a finite dimensional $k$-vector space?]

    See Hartshorne\cite{ha} section 2.5 Theorem 5.19. We should show $X$ is a projective scheme over $k$ and $L$ is a coherent $\calo_X$-module to apply the theorem 5.19.

    Since we assume $X$ is a projective variety which is equivalent to projective integral scheme over $k$ by Hartshorne\cite{ha} section 2.4 Prop 4.10, so $X$ is a projective scheme.

    Since $\calo_{\P^{m_1}}(1)$ is an invertible sheaf on $\P^{m_1}$, so $L$ is an invertible sheaf on $X$. Thus $L$ is a coherent $\calo_X$-module.
    \item[What is a $\phi$?]

    \textblue{It remains to be learned.}
  \end{description}
\end{rem}
