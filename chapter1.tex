\bfsection{1 Some basics of algebraic number theory}

\bfsubsection{Lemma 1.3}
\barquo{
Recall that $\trform{\;,\;}$ is non-degenerate if the Gramm matrix with respect to one (and hence any) basis of $L$ over $F$ is invertible.
}
\begin{proof}
  Almost trivial. Try to prove it.
\end{proof}


\bfsubsection{Proposition 1.4}
\barquo{
Let $\{ \beta_1, \cdots , \beta_n \}$ be the dual basis of $\{ \gra_1, \cdots , \gra_n \} $ with respect to $\trform{ \; , \; }$. Then, for any $x \in O_K$, we have $x = \trform{x,\gra_1} \beta_1 + \cdots + \trform{x,\gra_n} \beta_n$.
}
\begin{proof}
Since the trace form $\trform{\;,\;}$ is nondegenerate, $K \to K^* \st x \mapsto \trform{ \cdot, x}$ is a isomorphism. Let $p_i \colon K \to \Q$ be a projection map such that $p_i(x_1\gra_1 + \cdots + x_n\gra_n) = x_i$. Then, we set $\beta_j$ the preimage of $p_j$.
  \begin{comment}
   $\trform{\;, \gra_i}$ are linearly independent in $\Hom_{\Q} (K,\Q) = K^*$ and form $\Q$-basis of $K^*$.

  There are $\beta_{ij} \in \Q$ such that
  \[
  p_i = \sum_{j=1}^n \trform{\;, \gra_j} \beta_{ij}.
  \]
  This means $id_K = \sum_i \gra_i p_i = \sum_j \trform{\;, \gra_j} \sum_i \gra_i \beta_{ij}$, then we get
  $O_K \subset \Z \beta_1 + \cdots + \Z \beta_n$ for $\beta_j = \sum_i \gra_i \beta_{ij}$. Since $id_K  = \sum_j \trform{\;, \gra_j} \beta_j$ , $\beta_j$ are basis of $K$ and $\Z \beta_1 + \cdots + \Z \beta_n$ is a free $\Z$-module.

  We set $c_{ij} = \trform{\gra_i, \gra_j}$. And we get
  \[
  \grd_{ik} = p_i(\gra_k) = \sum_j \beta_{ij} c_{jk}.
  \]
  That means $I = \beta c$ by setting $\beta = (\beta_{ij}), c=(c_{ij})$, so $\beta$ is symmetric i.e. $\beta_{ij} = \beta_{ji}$.

  Then, we get
  \begin{align*}
    \trform{\beta_j, \gra_k} &= \sum_i \beta_{ij} \trform{\gra_i, \gra_k} \\
    &= \sum_i \beta_{ji} \trform{\gra_i, \gra_k} \\
    &= p_j(\gra_k ) \\
    &= \grd_{jk}.
  \end{align*}
  This is suggestive of orthogonality.
\end{comment}
\end{proof}



\bfsubsection{Lemma 1.7}
\barquo{
To see this, we take $t \in P(O_K)_P$ with $t \notin P^2(O_K)_P$.
}
\begin{rem}
  From Nakayama's lemma.
\end{rem}


\bfsubsection{Adjacent to Lemma 1.8}
\barquo{
For a nonzero prime ideal $P$ of $O_K$, we set $P \cap \Z = (p)$, where $p$ is a prime of $Z$. Because $O_K$ is a free $Z$-module of rank $[K:\Q]$, $O_K/P$ is a finite extension of $\Z/(p)$ with degree at most $[K:\Q]$.
}
\begin{proof}
There is a canonical surjection $O_K/pO_K \to O_K /P$, so we get $\# (O_K /P) \leq \# (O_K/pO_K)$. But we obtain $O_K/pO_K \cong O_K \ts_{\Z} \Z / p \Z$. Since $O_K$ is a free $\Z$-module of rank $n = [K:\Q]$, we conclude $O_K/pO_K \cong (\Z / p \Z )^n $. So, $\# (O_K /P) \leq \# (O_K/pO_K) = p^n$.
\end{proof}



\bfsubsection{Lemma 1.8}
\barquo{
\[
\bigoplus_{i=1}^r O_K/P_i^{e_i} = \bigoplus_{i=1}^r (O_K/P_i^{e_i})_{P_i}
\]
}
\begin{proof}
  Because $O_K/P_i^{e_i}$ is a local ring with maximal ideal $P_i/P_i^{e_i}$.
\end{proof}


\bfsubsection{Adjacent to Theorem 1.9}
\barquo{
we consider the value $\sqrt{\det (\kakko{e_i,e_j})}$.
}
\begin{rem}
  Why we get $\det (\kakko{e_i,e_j})$? Apply Gram-Schmidt orthonormalization.
\end{rem}


\bfsubsection{Adjacent to Theorem 1.9}
\barquo{
Then $\vol (M, \kakko{,} )$ is equal to the volume of the $n$-dimensional parallelpiped $\Pi$ spanned by $e_1, \cdots , e_n$,
}
\begin{proof}
  Let $F \colon (V, \kakko{,}) \to \R^n$ be  an isometric isomorphism. Then, we generate
  \begin{align*}
    \vol (M, \kakko{,} )^2 &= \det (\kakko{e_i,e_j}) \\
    &= \det (\kakko{Fe_i,Fe_j})
  \end{align*}
  We set $E = (Ee_1, \cdots, Fe_n)$. $E \in M_n(\R)$. Then we get $( \kakko{Fe_i,Fe_j})_{i,j} = {}^tE E$, and $\vol (M, \kakko{,} ) = \abs{\det E }$. From Yukie\cite{雪江線形} Theorem 4.9.1, $\abs{\det E } = \vol(\Pi)$.
\end{proof}



\bfsubsection{Proposition 1.11}
\barquo{
The form $\kakko{,}_K$ is an inner product on $V$.
}
\begin{rem}
  $\kakko{,}_K$ is trivially an inner product on $K$. Why should we show this?

  Let $S$ be a $\Q$ vector space and $\kakko{,}$ a inner product on $S$. Then, bilinear form extended to $S \ts_{\Q} \R$ may not be an inner product. For example, set $S = \Q[\sqrt{2}]$ and $\kakko{x,y} = xy$. 
\end{rem}



\bfsubsection{Lemma 1.12}
\barquo{
$\# (O_K/I)$ is finite. Then $I$ is a free $\Z$-module of rank $n$.
}
\begin{proof}
  $I \subset O_K$ is a free $\Z$-module. Since $\# (O_K/I)$ is finite, we get $\forall x \in K \; \exists n \in \Z \st nx \in I$. So we obtain $I \ts_{\Z} \Q = K$. The rank of $I$ is $n$.
\end{proof}

\bfsubsection{Lemma 1.16}
\barquo{
Because $(O_K)_P$ is a principal ideal domain, $(O_{K'})_P$ is a free $(O_K)_P$-module of rank $[K':K]$.
}
\begin{proof}
See the proof of Prop 1.4. We obtain $O_{K'} \subset O_K \beta_1 \oplus \cdots \oplus  O_K \beta_n$ for some $\beta_i \in K'$. Taking a localization, we get $(O_{K'})_P \subset (O_K)_P \beta_1 \oplus \cdots \oplus  (O_K)_P \beta_n$. Since $(O_K)_P$ is a PID, $(O_{K'})_P$ is a free $(O_K)_P$-module. The rank is not lower than $[K':K]$ because integral basis generate $K'$ over $K$.
\end{proof}


\bfsubsection{Lemma 1.16}
\barquo{
Thus
\begin{align*}
  \dim_{O_K/P} O_{K'}/PO_{K'} &=   \dim_{O_K/P} (O_{K'})_P / P(O_{K'})_P \\
  &=  \dim_{O_K/P} ((O_{K})_P / P(O_{K})_P) \ts_{(O_K)_P} (O_{K'})_P
\end{align*}
}
\begin{proof}
  We set $A = O_K, A'=O_{K'}$. Then we get
  \begin{align*}
    A'/PA' &\cong A' \ts_A A/P \\
    &\cong A' \ts_A \Frac A/P \\
    &\cong A' \ts_A \Coker (PA_P \to A_P) \\
    &\cong \Coker (A' \ts_A PA_P \to A' \ts_A A_P) \\
    &\cong (A')_P / P(A')_P \\
    (A')_P / P(A')_P  &\cong A' \ts_A \Coker (PA_P \to A_P) \\
    &\cong A' \ts_A A_P / PA_P \\
    &\cong (A' \ts_A  A_P) \ts_{A_P} A_P / PA_P \\
    &\cong (A')_P \ts_{A_P} A_P / PA_P.
  \end{align*}
\end{proof}



\bfsubsection{Adjacent to Lemma 1.17}
\barquo{
We take a integral basis $\{ \gro_1, \cdots , \gro_n \}$ of $O_K$, we denote by $\{ \beta_1, \cdots, \beta_n  \}$ the dual basis with respect to $\trform{\;,\;}$. Then we have $\calm = \Z \beta_1 + \cdots +  \Z \beta_n$.
}
\begin{proof}
  See the note of Prop 1.4.
\end{proof}


\bfsubsection{Adjacent to Lemma 1.17}
\barquo{
The difference of $K$ is defined by $\cald_K = \calm^{-1}$. Because $O_K \subset \calm$, we have $\cald_K \subset O_K$, so $\cald_K$ is an ideal of $O_K$.
}
\begin{proof}
  $
  O_K = \calm \calm^{-1} = \cald_K \calm \supset \cald_K O_K \supset \cald_K.
  $
\end{proof}



\bfsubsection{Lemma 1.17}
\barquo{
Indeed, because $\#(O_K / \cald_K) = \# (\calm / O_K)$,
}
\begin{proof}
  See Yukie\cite{雪江N2} Proposition 1.8.6.
\end{proof}


\bfsubsection{Theorem 1.18}
\barquo{
Lemma 1.17 (3) gives
\[
\log_p (\# ( ( (O_K)_P / (\cald_K)_P ) ) = \sum_i \ord_{P_i} (\cald_K)_{f_i}
\]
}
\begin{proof}
  \textblue{It remains to be solved.}
\end{proof}


\bfsubsection{Theorem 1.18}
\barquo{
Because $\# (O_K / \cald_K) = \prod_{p \in S} \# ( ( (O_K)_P / (\cald_K)_P ) $, we obtain the assertion.
}
\begin{proof}
  See Yukie\cite{雪江N2} Prop1.8.9.
\end{proof}
