\bfsection{Some basics of algebraic number theory}

\bfsubsection{Lemma 1.3}
\barquo{
Recall that $\trform{\;,\;}$ is non-degenerate if the Gramm matrix with respect to one (and hence any) basis of $L$ over $F$ is invertible.
}
\begin{proof}
  Almost trivial. Try to prove it.
\end{proof}


\bfsubsection{Proposition 1.4}
\barquo{
Let $\{ \beta_1, \cdots , \beta_n \}$ be the dual basis of $\{ \gra_1, \cdots , \gra_n \} $ with respect to $\trform{ \; , \; }$. Then, for any $x \in O_K$, we have $x = \trform{x,\gra_1} \beta_1 + \cdots + \trform{x,\gra_n} \beta_n$.
}
\begin{proof}
  Since the trace form $\trform{\;,\;}$ is degenerate, $\trform{\;, \gra_i}$ are linearly independent in $\Hom_{\Q} (K,\Q) = K^*$ and form $\Q$-basis of $K^*$.

  Let $p_i \colon K \to \Q$ be a projection map such that $p_i(x_1\gra_1 + \cdots + x_n\gra_n) = x_i$. There are $\beta_{ij} \in \Q$ such that
  \[
  p_i = \sum_{j=1}^n \trform{\;, \gra_j} \beta_{ij}.
  \]
  This means $id_K = \sum_i \gra_i p_i = \sum_j \trform{\;, \gra_j} \sum_i \gra_i \beta_{ij}$, then we get
  $O_K \subset \Z \beta_1 + \cdots + \Z \beta_n$ for $\beta_j = \sum_i \gra_i \beta_{ij}$. Since $id_K  = \sum_j \trform{\;, \gra_j} \beta_j$ , $\beta_j$ are basis of $K$ and $\Z \beta_1 + \cdots + \Z \beta_n$ is a free $\Z$-module.

  We set $c_{ij} = \trform{\gra_i, \gra_j}$. And we get
  \[
  \grd_{ik} = p_i(\gra_k) = \sum_j \beta_{ij} c_{jk}.
  \]
  That means $I = \beta c$ by setting $\beta = (\beta_{ij}), c=(c_{ij})$, so $\beta$ is symmetric i.e. $\beta_{ij} = \beta_{ji}$.

  Then, we get
  \begin{align*}
    \trform{\beta_j, \gra_k} &= \sum_i \beta_{ij} \trform{\gra_i, \gra_k} \\
    &= \sum_i \beta_{ji} \trform{\gra_i, \gra_k} \\
    &= p_j(\gra_k ) \\
    &= \grd_{jk}.
  \end{align*}
  This is suggestive of orthogonality.
\end{proof}


\bfsubsection{Lemma 1.16}
\barquo{
Because $(O_K)_P$ is a principal ideal domain, $(O_{K'})_P$ is a free $(O_K)_P$-module of rank $[K':K]$.
}
\begin{proof}
It remains to be answered.
\end{proof}


\bfsubsection{Lemma 1.16}
\barquo{
Thus
\begin{align*}
  \dim_{O_K/P} O_{K'}/PO_{K'} &=   \dim_{O_K/P} (O_{K'})_P / P(O_{K'})_P \\
  &=  \dim_{O_K/P} ((O_{K})_P / P(O_{K})_P) \ts_{(O_K)_P} (O_{K'})_P
\end{align*}
}
\begin{proof}
  We set $A = O_K, A'=O_{K'}$. Then we get
  \begin{align*}
    A'/PA' &\cong A' \ts_A A/P \\
    &\cong A' \ts_A \Frac A/P \\
    &\cong A' \ts_A \Coker (PA_P \to A_P) \\
    &\cong \Coker (A' \ts_A PA_P \to A' \ts_A A_P) \\
    &\cong (A')_P / P(A')_P \\
    (A')_P / P(A')_P  &\cong A' \ts_A \Coker (PA_P \to A_P) \\
    &\cong A' \ts_A A_P / PA_P \\
    &\cong (A' \ts_A  A_P) \ts_{A_P} A_P / PA_P \\
    &\cong (A')_P \ts_{A_P} A_P / PA_P.
  \end{align*}
\end{proof}



\bfsubsection{Adjacent to Lemma 1.17}
\barquo{
We take a integral basis $\{ \gro_1, \cdots , \gro_n \}$ of $O_K$, we denote by $\{ \beta_1, \cdots, \beta_n  \}$ the dual basis with respect to $\trform{\;,\;}$. Then we have $\calm = \Z \beta_1 + \cdots +  \Z \beta_n$.
}
\begin{proof}
  See the note of Prop 1.4.
\end{proof}


\bfsubsection{Adjacent to Lemma 1.17}
\barquo{
The difference of $K$ is defined by $\cald_K = \calm^{-1}$. Because $O_K \subset \calm$, we have $\cald_K \subset O_K$, so $\cald_K$ is an ideal of $O_K$.
}
\begin{proof}
  $
  O_K = \calm \calm^{-1} = \cald_K \calm \supset \cald_K O_K \supset \cald_K.
  $
\end{proof}



\bfsubsection{Lemma 1.17}
\barquo{
Indeed, because $\#(O_K / \cald_K) = \# (\calm / O_K)$,
}
\begin{proof}
  See Yukie\cite{雪江N2} Proposition 1.8.6.
\end{proof}




\bfsubsection{Adjacent to 2.4}
\barquo{
In other words, the absolute (logarithmic) Weil height is not invariant under linear coordinate changes. This is why it is sometimes called the naive height.
}
\begin{que}
  Why they want the height to be invariant? Is there non-constant invariant height?
\end{que}
