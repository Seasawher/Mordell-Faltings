\bfsection{1 Some basics of algebraic number theory}

\bfsubsection{Lemma 1.3}
\barquo{
Recall that $\trform{\;,\;}$ is non-degenerate if the Gramm matrix with respect to one (and hence any) basis of $L$ over $F$ is invertible.
}
\begin{proof}
  Almost trivial. Try to prove it.
\end{proof}


\bfsubsection{Proposition 1.4}
\barquo{
Let $\{ \beta_1, \cdots , \beta_n \}$ be the dual basis of $\{ \gra_1, \cdots , \gra_n \} $ with respect to $\trform{ \; , \; }$. Then, for any $x \in O_K$, we have $x = \trform{x,\gra_1} \beta_1 + \cdots + \trform{x,\gra_n} \beta_n$.
}
\begin{proof}
Since the trace form $\trform{\;,\;}$ is nondegenerate, $K \to K^* \st x \mapsto \trform{ \cdot, x}$ is a isomorphism. Let $p_i \colon K \to \Q$ be a projection map such that $p_i(x_1\gra_1 + \cdots + x_n\gra_n) = x_i$. Then, we set $\beta_j$ the preimage of $p_j$.

\end{proof}



\bfsubsection{Lemma 1.7}
\barquo{
To see this, we take $t \in P(O_K)_P$ with $t \notin P^2(O_K)_P$.
}
\begin{rem}
  From Nakayama's lemma.
\end{rem}


\bfsubsection{Adjacent to Lemma 1.8}
\barquo{
For a nonzero prime ideal $P$ of $O_K$, we set $P \cap \Z = (p)$, where $p$ is a prime of $Z$. Because $O_K$ is a free $Z$-module of rank $[K:\Q]$, $O_K/P$ is a finite extension of $\Z/(p)$ with degree at most $[K:\Q]$.
}
\begin{proof}
There is a canonical surjection $O_K/pO_K \to O_K /P$, so we get $\# (O_K /P) \leq \# (O_K/pO_K)$. But we obtain $O_K/pO_K \cong O_K \ts_{\Z} \Z / p \Z$. Since $O_K$ is a free $\Z$-module of rank $n = [K:\Q]$, we conclude $O_K/pO_K \cong (\Z / p \Z )^n $. So, $\# (O_K /P) \leq \# (O_K/pO_K) = p^n$.
\end{proof}



\bfsubsection{Lemma 1.8}
\barquo{
\[
\bigoplus_{i=1}^r O_K/P_i^{e_i} = \bigoplus_{i=1}^r (O_K/P_i^{e_i})_{P_i}
\]
}
\begin{proof}
  Because $O_K/P_i^{e_i}$ is a local ring with maximal ideal $P_i/P_i^{e_i}$.
\end{proof}


\bfsubsection{Adjacent to Theorem 1.9}
\barquo{
we consider the value $\sqrt{\det (\kakko{e_i,e_j})}$.
}
\begin{rem}
  Why we get $\det (\kakko{e_i,e_j})$? Apply Gram-Schmidt orthonormalization.
\end{rem}


\bfsubsection{Adjacent to Theorem 1.9}
\barquo{
Then $\vol (M, \kakko{,} )$ is equal to the volume of the $n$-dimensional parallelpiped $\Pi$ spanned by $e_1, \cdots , e_n$,
}
\begin{proof}
  Let $F \colon (V, \kakko{,}) \to \R^n$ be  an isometric isomorphism. Then, we generate
  \begin{align*}
    \vol (M, \kakko{,} )^2 &= \det (\kakko{e_i,e_j}) \\
    &= \det (\kakko{Fe_i,Fe_j})
  \end{align*}
  We set $E = (Ee_1, \cdots, Fe_n)$. $E \in M_n(\R)$. Then we get $( \kakko{Fe_i,Fe_j})_{i,j} = {}^tE E$, and $\vol (M, \kakko{,} ) = \abs{\det E }$. From Yukie\cite{雪江線形} Theorem 4.9.1, $\abs{\det E } = \vol(\Pi)$.
\end{proof}



\bfsubsection{Proposition 1.11}
\barquo{
The form $\kakko{,}_K$ is an inner product on $V$.
}
\begin{rem}
  $\kakko{,}_K$ is trivially an inner product on $K$. Why should we show this?

  Let $S$ be a $\Q$ vector space and $\kakko{,}$ a inner product on $S$. Then, bilinear form extended to $S \ts_{\Q} \R$ may not be an inner product. For example, set $S = \Q[\sqrt{2}]$ and $\kakko{x,y} = xy$.
\end{rem}



\bfsubsection{Lemma 1.12}
\barquo{
$\# (O_K/I)$ is finite. Then $I$ is a free $\Z$-module of rank $n$.
}
\begin{proof}
  $I \subset O_K$ is a free $\Z$-module. Since $\# (O_K/I)$ is finite, we get $\forall x \in K \; \exists n \in \Z \st nx \in I$. So we obtain $I \ts_{\Z} \Q = K$. The rank of $I$ is $n$.
\end{proof}

\bfsubsection{Lemma 1.16}
\barquo{
We have $[K':K] = e_1 f_1 + \cdots + e_r f_r$.
}
\begin{proof}
See the proof of Prop 1.4. We obtain $O_{K'} \subset O_K \beta_1 \oplus \cdots \oplus  O_K \beta_n$ for some $\beta_i  \in K'$. That implies there is an injection such that $O_{K'} \to \bigoplus_i O_K$. Because localization is a flat module, we get $(O_{K'})_P \subset (O_K )_P \beta_1 \oplus \cdots \oplus  (O_K )_P \beta_n$. Since $(O_K)_P$ is a PID, $(O_{K'})_P$ is a free $(O_K)_P$-module.
The rank is $[K':K]$ because
\[
(O_{K'})_P \ts_{(O_K)_P} K = (O_{K'} \ts_{O_K} (O_K)_P ) \ts_{(O_K)_P}  K = O_{K'} \ts_{O_K} K = K'.
\]
Thus, as a $O_K /  P$ module,
\begin{align*}
O_{K'} / P O_{K'} &\cong O_K/P \ts_{O_K} O_{K'} \\
&\cong  (O_K/P \ts_{O_K} (O_K)_P \ts_{(O_K)_P}  (O_K)_P ) \ts_{O_K}  O_{K'}  \\
&\cong (O_K/P \ts_{O_K} (O_K)_P) \ts_{(O_K)_P} ( O_{K'})_P \\
&\cong \bigoplus_{[K':K]}  (O_K/P \ts_{O_K} (O_K)_P) \\
&\cong \bigoplus_{[K':K]}  O_K /P .
\end{align*}
Then it follows that
\begin{align*}
  \#(O_K / P)^{[K':K]} &= \# (O_{K'} / P O_{K'} ) \\
  &= \prod_i \#(O_{K'} / {P'_i}^{e_i} ) \\
  &=  \prod_i \#(O_{K'} / P'_i )^{e_i} \\
  &=  \prod_i \#(O_{K} / P)^{e_{i} f_{i} }.
\end{align*}
Thus $[K':K] = \sum_i e_{i} f_{i}$.
\end{proof}

\begin{comment}
\bfsubsection{Lemma 1.16}
\barquo{
Thus
\begin{align*}
  \dim_{O_K/P} O_{K'}/PO_{K'} &=   \dim_{O_K/P} (O_{K'})_P / P(O_{K'})_P \\
  &=  \dim_{O_K/P} ((O_{K})_P / P(O_{K})_P) \ts_{(O_K)_P} (O_{K'})_P
\end{align*}
}
\begin{proof}
  We set $A = O_K, A'=O_{K'}$. Then we get
  \begin{align*}
    A'/PA' &\cong A' \ts_A A/P \\
    &\cong A' \ts_A \Frac A/P \\
    &\cong A' \ts_A \Coker (PA_P \to A_P) \\
    &\cong \Coker (A' \ts_A PA_P \to A' \ts_A A_P) \\
    &\cong (A')_P / P(A')_P \\
    (A')_P / P(A')_P  &\cong A' \ts_A \Coker (PA_P \to A_P) \\
    &\cong A' \ts_A A_P / PA_P \\
    &\cong (A' \ts_A  A_P) \ts_{A_P} A_P / PA_P \\
    &\cong (A')_P \ts_{A_P} A_P / PA_P.
  \end{align*}
\end{proof}
\end{comment}

\bfsubsection{Adjacent to Lemma 1.17}
\barquo{
We take a integral basis $\{ \gro_1, \cdots , \gro_n \}$ of $O_K$, we denote by $\{ \beta_1, \cdots, \beta_n  \}$ the dual basis with respect to $\trform{\;,\;}$. Then we have $\calm = \Z \beta_1 + \cdots +  \Z \beta_n$.
}
\begin{proof}
  See the note of Prop 1.4.
\end{proof}





\bfsubsection{Lemma 1.17}
\barquo{
Indeed, because $\#(O_K / \cald_K) = \# (\calm / O_K)$,
}
\begin{proof}
  See Yukie\cite{雪江N2} Proposition 1.8.6.
\end{proof}








\bfsubsection{Theorem 1.18}
\barquo{
Then we have
\[
\abs{ D_{K/\Q}} \leq \prod_{p \in S} p^{n-1+n \log_p(n)}.
\]
}
\begin{proof}
  We may assume that $S = \setmid{p \in \Z}{\text{$p$ is ramified} }$. Set $B = O_K$ and $I = D_K$.
\begin{description}
  \item[Step 1] Let $p \in \Z$ be a prime number. Then $B_p$ and $I_p$ are free $\Z_p$-module of rank $n$. So there is a matrix $C \in M_n(\Z_p) \cap GL_n(\Q_p)$ such that the following diagram
  \[
  \xymatrix{
  I_p \ar[r] \ar[d]  & B_p \ar[d] \\
  \Z_p^n \ar[r]^-C & \Z_p^n
  }
  \]
  commute. Then
  \begin{align*}
    \#(B/I \ts_{\Z} \Z_p) &= \#(\Coker C) \\
    &= \#(\Z_p / (\det C) \Z_p) \\
    &= \#(\wh{\Z}_p / (\det C) \wh{\Z}_p) &(\text{See Yukie\cite{雪江N2} Proposition 1.2.13}) \\
    &= \#(B/I \ts_{\Z} \wh{\Z}_p).
  \end{align*}
\item[Step 2] It follows that
  \begin{align*}
    B/I \ts_{\Z} \wh{\Z}_p &\cong   B/I \ts_B B \ts_{\Z} \wh{\Z}_p \\
    &\cong   B/I \ts_B  \bigoplus_i \wh{B}_{P_i} &(\text{See Yukie\cite{雪江N2} Theorem 1.3.23 }) \\
    &\cong \bigoplus_i \wh{B}_{P_i} / P_i^{\ord_{P_i}(I)} \wh{B}_{P_i} \\
    &\cong \bigoplus_i B/P_i^{\ord_{P_i}(I) }
    \end{align*}
\item[Step 3] Set $J = I \cap \Z$. Because $B/I$ is finitely generated $\Z$-module, we get
\[
\Supp_{\Z} (B/I) = V(\ann_{\Z} (B/I)) = V(J).
\]
See Matsumura\cite{松村} adjacent to Theorem 4.4 if you do not understand the first equation.

And for any prime number $p \in \Z$, then we obtain
\begin{align*}
  p \not\in \Supp_{\Z}(B/I) &\iff B/I \ts_{\Z} \Z_p = 0 \\
  &\iff \#(B/I \ts_{\Z} \Z_p ) = 1 \\
  &\iff \forall i \; \#(B  / P_i^{\ord_{P_i}(I) }) = 1 \\
  &\iff \ord_{P_i}(I) = 0 \\
  &\iff \text{$p$ is unramified}
  \end{align*}
  Thus we conclude $V(J) = \Supp_{\Z} (B/I) = S$.
  \item[Step 4] Then we get
  \begin{align*}
    \#(B/I \ts_{\Z} \Z_p ) &= \prod_i \#( B/P_i^{\ord_{P_i}(I) } )  \\
    &= \prod_i \#( B/P_i)^{\ord_{P_i}(I) }   \\
    &= \prod_i \#( \Z/p)^{f_i \ord_{P_i}(I) }.
  \end{align*}
  So we conclude $\log_p( \#(B/I \ts_{\Z} \Z_p ) ) \leq n - 1 + n \log_p(n)$.
  \item[Step 5] Recall that $J = \ann_{\Z}(B/I)$. Then we get
  \begin{align*}
    B/I &\cong (B/I )/ J (B/I) \\
    &\cong \bigoplus_{p \in S} (B/I) / p^e (B/I) &(\text{$e$ depends on $p$}) \\
    &\cong \bigoplus_{p \in S} B /( p^e B+ I) \\
    &\cong \bigoplus_{p \in S} B /( p^e B+ I) \ts_{\Z} \Z_p \\
    &\cong \bigoplus_{p \in S} B_p /( p^e B_p+ I_p) \\
        &\cong \bigoplus_{p \in S} B_p /( J B_p+ I_p) \\
        &\cong \bigoplus_{p \in S} B_p / I_p  \\
  \end{align*}
  Now we conclude that
  \[
  \abs{ D_{K/\Q}} = \#(B/I) = \prod_{p \in S} \#(B_p/I_p) \leq \prod_{p \in S} p^{n-1+n\log_p(n)}.
  \]
\end{description}
\end{proof}
